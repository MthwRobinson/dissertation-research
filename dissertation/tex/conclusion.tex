\doublespacing
\chapter{Conclusion and Future Work} \label{chap:conclusion}

\section{Summary}

This work employed regression models to analyze the impact of network structure and requirements crowdsourcing on the performance of OSS projects with respect to six measures of effectiveness. The results support the hypotheses outlined in Chapter \ref{chap:contribution} and demonstrate that:
\begin{enumerate}
    \item Stakeholder networks with concentrated hub-and-spoke structures have stronger project management outcomes
    \item Requirements crowdsourcing faces diminishing marginal returns
    \item The impact of requirements crowdsourcing changes with the structure of the stakeholder networks
\end{enumerate}

Based on these results, OSS project managers would benefit from instituting policies to encourage additional crowd engagement for projects that current source a small percentage of requirements from the crowd, and employing CrowdRE techniques to assist with triaging and prioritizing requirements when the share of crowdsourced requirements rises.

Requirement close-out time and requirement response time indicate the level of engagement and productivity of the project team. The regression results show that, under most circumstances, crowdsourcing a greater share of requirements increases the amount of time it takes to close out requirements. Figure \ref{reqs_contributors_over_time} demonstrates the reason for this phenomenon: the volume of open requirements increases faster than the size of a project team over time, leading to a growing backlog of unaddressed requirements. OSS projects whose stakeholder networks have hub-and-spoke structures better contend with this challenge, with requirement close-out time remaining close to flat under the most ideal circumstances.

While crowdsourcing always increases close-out time, the effect on requirement response time varies based on the stakeholder network structure and the current share of crowdsourced requirements. For projects that crowdsource a small share of requirements, and increase in the proportion of crowdsourced requirements decreases requirement response time. Conversely, additional crowdsourcing increases requirement response time for OSS projects that currently crowdsource a high proportion of requirements. The crossover point occurs at the 30-70\% mark, depending on the stakeholder networks structure, and occurs later for more concentrated stakeholder networks. This fortifies the notion that OSS projects benefit from having organized and concentrated stakeholder networks and lends support to the micro-crowd strategy Levy, Hadar, and Te'eni~\cite{levy} propose for building engagement from small, highly connected groups of users.

Requirement comment activity, contributor retention time, and requirements per crowd member measure levels of crowd engagement. Comment activity follows an unusual pattern in that additional requirements crowdsourcing lowers comment activity for OSS projects that currently source a very small or very large proportion of requirements from the crowd. For OSS projects that source a moderate share of requirements from the crowd, crowdsourcing additional requirements decreases comment activity, except for highly concentrated networks. For these networks, the marginal effect on comment activity decreases at moderate proportions of requirements crowdsourcing, but never turns negative. Again, this supports the argument that more concentrated networks promote better performance.

Increasing the share of crowdsourced requirements reduces the average retention time of crowd members. This holds at all values for the current share of crowdsourced requirements, although the effect decreases as the share of crowdsourced requirements increases. While OSS projects with less dispersed stakeholder networks observe a smaller decrease in retention time, the difference disappears for projects that source more than 40\% of requirements from the crowd. For OSS projects that currently crowdsource a small share of requirements from the crowd, an increase in crowdsourcing reduces the average requirements per crowd member. The average requirements per crowd member, however, increases with the proportion of crowdsourced requirements for OSS projects that already generate most of their requirements from the crowd. Taken together, the results for retention time and requirements per crowd member lead to the conclusion that, for OSS projects that crowdsource a large proportion of requirements, most of the requirements originate from a small set of highly active crowd members. 

Finally, the results for requirements volume show that increases in the share of crowdsourced requirements results in a lower volume of requirements for OSS projects that generate a limited number of requirements from the crowd. For OSS projects that crowdsource a substantial portion of requirements, the effect flips, meaning that crowdsourcing a greater share of requirements results in a higher requirement volume. Stakeholder networks with higher levels of concentration tend to have higher requirement volumes. Since increasing the volume of requirements does not reflect an unambiguously good or bad outcome, trade-off analysis enables OSS project managers to understand the impact of encouraging the crowd to submit more requirements.

\section{Discussion}

This research adds to the existing literature in crowdsourcing and CrowdRE, for several reasons. First, the analysis presents a novel application of CrowdRE within an OSS context. In his review of the state of CrowdRE, Glinz~\cite{glinz} highlights the lack of CrowdRE applications for OSS requirements generation as a research gap. This research both utilizes CrowdRE to assess the impact of stakeholder network structure and requirements crowdsourcing on OSS project performance and develops a CrowdRE approach OSS project managers can use to determine the best strategy for engaging with the crowd.

Second, the results suggest that, beyond a certain threshold, crowdsourcing additional requirements can hurt an OSS project more than it helps. For OSS projects that already crowdsource more than about 70\% of requirements, increasing the share of crowdsourced requirements results in longer requirement close-out and response times, a growing backlog of unaddressed requirements, and shorter retention time for crowd members. These factors reinforce one another. A higher volume of requirements increases average requirement close-out time, which in turn lowers the crowd retention rate.

Third, this research suggests that OSS project managers need to consider not only how to incentive crowd participation, but also when. The discussion in the previous paragraph does not imply that project managers should discourage requirements crowdsourcing, even when they already crowdsource a sizable share of requirements. Rather, once an OSS project already crowdsources a substantial portion of requirements, project managers should focus instead on prioritizing existing requirements by applying the collaborative requirements elicitation techniques outlined in Section \ref{network_re}.

Fourth, this research shows that stakeholder network structure has a significant effect on outcomes for OSS projects, and that the impact of stakeholder network structure changes depending on the proportion of requirements a project crowdsources. The results show that more concentrated stakeholder networks perform better across multiple measures of effectiveness. Highly concentrated hub-and-spoke networks correlate with shorter requirement close-out and response times and higher comment activity on requirements. In addition to validating existing research by Iyer and Lyytinen~\cite{iyer} and Toral, et al.~\cite{toral}, this research expands the current understanding of the topic by showing that the effect of network structure changes with the share of crowdsourced requirements, utilizing multiple network structure variables, and considering additional measures of effectiveness.

Finally, this paper uses empirical analysis to verify several claims from the CrowdRE literature. First, the surprising conclusion that average comment activity increases for projects with a high share of crowdsourced requirements even as the total volume of requirements grows suggests that crowd engagement succeeds in surfacing new perspectives. Groen et al.~\cite{groen} claims gathering perspectives from a more diverse range of stakeholders as one of the main benefits of crowd engagement. Furthermore, the results showing lower retention when an OSS projects crowdsources a larger share of requirements validates concerns within the \mbox{CrowdRE} literature~\cite{snijders, snijders2, levy} about maintaining crowd motivation, and suggests applying CrowdRE techniques could help. Additionally, the observation that increases in requirement close-out times lower crowd retention supports Groen, et al.'s~\cite{groen} assertion that crowds often include impact seekers, whose motivation depends on the implementation of their suggestions. All of this reinforces the notion that the crowd represents an important construct for thinking about the requirements process for OSS projects.

This paper also has important practical implications. Most importantly, it suggests both that OSS project teams can benefit from using CrowdRE tools, and that the decision on which tool to use depends on the current state of the project. If projects that have stakeholder networks with hub-and-spoke structures produce better outcomes, project managers can benefit from implementing strategies to encourage the formation of these structures. Hub-and-spoke networks result from direct interaction between a crowd member and the appropriate member of the project team. Natural language processing tools such as Mobasher's~\cite{mobasher} recommendation system that identify the appropriate implementer based on the content of a requirement could help promote such connections. Assigning a project team member to engage with crowd members and route requirements could also result in more efficient patterns of engagement between team members and the crowd.

This work suggests that OSS project teams must employ different CrowdRE tools depending on the current level of requirements crowdsourcing for the project. If a project currently crowdsources a small share of requirements, OSS project managers would benefit from gamification techniques~\cite{snijders, snijders2} that incentivize additional crowd participation. Once a project crowdsources a substantial share of requirements, OSS project managers should focus on collaborative elicitation techniques for routing and prioritizing requirements.

\section{Limitations and Threats to Validity}
\label{limitations}

While this research develops useful insights, readers should keep in mind several limitations and threats to validity. The main limitations to this study stem from the fact that it focuses on stable, widely adopted OSS projects. Therefore, the results do not generalize to less active OSS software, proprietary software, or hardware systems. Threats to validity include that the study does not include every OSS measure of effectiveness from the literature, makes an assumption that all OSS projects benefit from the same network structure, and does not consider how stakeholder network structure changes over time. While these factors limit the generalizability of the study, the conclusions of the research are solidly supportable within the scope of widely adopted OSS projects.

\subsection{Limitations}

First, this research only considers software. Physical products present different challenges because they have longer feedback cycles and less mutable specifications. Moreover, while the data set covers a broad range of OSS projects, it only includes active, widely adopted packages. The results may not apply equally well to smaller OSS packages with less active user bases. Additionally, as Scacchi~\cite{scacchi} notes, proprietary software and corporate-led OSS projects follow more formal procedures for requirements gathering and development. Because of these differences, the results in this paper do not apply to software projects in more structured settings.

Second, the OSS projects in the data set target developers as the user base. This presents several challenges. On average, since they come from the software development community, OSS and GitHub users have unusually strong technical skills. Therefore, they can formulate technical requirements without needing a product manager as an intermediary. These conditions do not hold for consumer-facing OSS projects, such as web-applications. Users of consumer-facing tools more commonly express the need for a particular feature, which a product manager translates into a technical requirement that the development team can implement. Due to the specialized nature of the crowds in the data set, the results in this paper do not generalized beyond OSS projects with a technical audience.

Third, this research only considers requirements elicitation for existing software. Consequently, the software projects in the data set have already advanced beyond the design phase. Formulating requirements for new software involves more uncertainty and necessarily benefits less from crowd participation because software does not have any users at the onset. Furthermore, crowd members have less incentive to participate in the requirements process because they do not yet actively rely on the software as a dependency. Therefore, the results in this research only apply to OSS projects that have advanced beyond the design phase.

Fourth, the CrowdRE literature typically considers a selective, well-moderated crowd~\cite{stakerare}. Research by Lim~\cite{stakerare, stakesource, lim} and others which indicates that crowd engagement results in higher quality requirements does not extend automatically to unmoderated crowds of OSS contributors. Assessing the impact of unmoderated crowd contributions on requirement quality necessitates further research.

Fifth, this research only considers the requirements generation process. It does not consider other components of the requirements process, such as validation and prioritization. As such, this research does not shed additional light on Lim, et al.'s~\cite{stakenet} conclusion that a high volume of crowd-generated requirements biases prioritization in favor of more active stakeholders.

Finally, the analysis does not capture every measure of effectiveness for OSS projects in the existing literature. Specifically, this study does not include product quality~\cite{stewart}, project team satisfaction~\cite{ghapanchi}, user satisfaction~\cite{ghapanchi}, requirement quality~\cite{ma}, and the extent to which the crowd agrees about the content and relatively importance of requirements~\cite{ma}. 

\subsection{Threats to Validity}

This research also faces a number of threats to validity. First, this research does not account for project complexity. Project complexity could impact the results because more complex projects increase the amount of time team members need to implement requirements and create a barrier to entry for crowd members to generate reasonably high quality requirements. Factors such as the number of lines of code, the number of modules, and the extent to which the modules interact within one another could serve as a proxy for project complexity

Second, this research assumes that different types of software projects have the same optimal network structure. In reality, low-level systems software might require higher levels of specialized expertise, and therefore benefit from a different patterns of crowd interaction than a web-development framework, which has a broader user base. Additionally, for more specialized software, the limited number of crowd members capable of contributing well-formed requirements may mitigate some of the negative effects that result from crowdsourcing too many requirements. More specialized projects, for example, may not suffer from an unwieldy backlog of crowd-generated requirements simply due to the fact that only a small number of crowd members have the expertise necessary to contribute requirements. Under these circumstances, requirements crowdsourcing may not face diminishing marginal returns.

Finally, the research assumes static stakeholder networks. In reality, stakeholder network structure changes over time as new developers join or leave. A core developer leaving a project could alter the structure of the stakeholder network quite dramatically. This presents less of a concern for this data set than for OSS projects more broadly because the data consists of a curated set of active projects. However, accounting for temporal dynamics in the stakeholder networks would make the analysis more robust and more broadly applicable.

\section{Future Research}

This research prompts several questions that could serve as the basis for follow-on research. First, this study shows that longer requirement close-out times lead to greater crowd attrition in the aggregate. Future research could evaluate this phenomenon at the individual contributor level by observing whether or not the implementation of a crowd member's requirement impacts the probability that the crowd member submits another requirement in the future.

Second, the results in this paper indicate that more concentrated stakeholder networks produce better outcomes for OSS projects. A reasonable hypothesis to explain this phenomenon is that each hub is a contributor who specializes in a particular aspect of the code base. These stakeholder networks may perform better as crowdsourcing expands because each contributor only needs to focus on a small subset of incoming requirements. An investigation to determine whether different hubs in these networks indeed specialize based on requirement type could yield valuable insights.

Third, one of the proposed benefits of CrowdRE is that it leads to the generation of new requirements that project managers would not have identified if they had not solicited feedback from the crowd. If crowd members and project team members produce similar requirements, that would negate this benefit. A research effort to determine the degree to which crowd-generated requirements differ from internally sourced requirements could help shed light on this issue.

Fourth, as suggested in Section \ref{trade-off}, future research could develop more sophisticated models to analyze the trade-off space between various measures of effectiveness. These models should account for the change in network structure that occurs when OSS projects source additional requirements from the crowd, and include sensitivity analysis that quantifies how a change in network structure would impact the trade-off space.

Fifth, future work could consider different types of crowds and requirements process activities. The results, for instance, may not hold for crowds with less technical acumen. Crowdsourcing for other requirements process activities, such as validation and prioritization, may likewise benefit from different patterns of engagement. 

Sixth, future research could consider aspects of an OSS project, such as requirement quality and project complexity, that did not factor into the analysis in this research. These variables present challenges due to the inability to directly measure them from the data. Measuring these variables would likely requiring training an additional predictive model, which would add uncertainty to the analysis.

Finally, this study considers stakeholder networks as a snapshot in time. In reality, the structure of these networks evolves along with their their performance characteristics. An effort to study how these stakeholder networks arrived at their current structure could help project managers design interventions that will guide their interactions with the crowd toward a positive outcome. Additional, future research could address how various types of developer interactions correlate with the evolution of network structure over time.

\section{Summary}

This study used a GitHub data set consisting of project management artifacts from a curated set of 562 open-source software projects to evaluate the impact of stakeholder network structure on the effectiveness of crowd-based requirements processes. The results supports each of the hypotheses in Section \ref{chap:contribution}. Specifically, the indicate that requirements crowdsourcing faces diminishing marginal returns, the impact of requirements crowdsourcing changes with stakeholder network structure, and more concentrated stakeholder networks promote stronger project management outcomes. As a consequence of these results, OSS project managers should use strategies such as gamification~\cite{dalpiaz} and micro-crowds~\cite{levy} to incentivize crowd participation when the share of crowdsourced requirements is low, and use CrowdRE to efficiently prioritize and execute on existing requirements when the share of crowdsourced requirements grows.

For four out of six measures of effectiveness, statistical analysis shows that the marginal impact of additional crowdsourcing changes with the structure of the stakeholder network and the current proportion of crowdsourced requirements. The results suggest that OSS projects whose stakeholder networks have hub-and-spoke structures perform better as the proportion of crowdsourced requirements increases. These structures result from direct engagement on requirements between crowd members and the appropriate member of the project team. OSS project managers can promote the development of stakeholder networks with hub-and-spoke structures by adopting CrowdRE strategies that route crowdsourced requirements based on their content.

The results also show that requirements crowdsourcing faces diminishing marginal returns as the proportion of crowdsourced requirements increases. In particular, crowdsourcing more than about 70\% of requirements results in longer requirement close-out and response times, a higher volume of open requirements, and shorter average retention times for crowd members. As such, beyond that threshold, OSS project members would benefit more from channeling existing crowd engagement than from incentivizing crowd members to submit additional requirements. While this research establishes a relationship between stakeholder network structure, the share of crowdsourced requirements, and measures of effectiveness for OSS projects, the results do not generalize to OSS projects with less technical users, proprietary software, or physical products. Overall, the outcome of this study points to the importance of stakeholder network structure to the effectiveness of the requirements process for OSS projects, confirm that the impact of requirements crowdsourcing depends on the specific circumstances of an OSS process, and suggests opportunities to employ CrowdRE to improve OSS project management outcomes.