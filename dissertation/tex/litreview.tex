\doublespacing
\chapter{Literature Review} \label{chap:litreview}

\section{Introduction}

The research in this dissertation spans and integrates three distinct fields: crowdsourcing, requirements engineering, and network analysis. The literature review chapter will first cover crowdsourcing definitions and taxonomies, and argue how CrowdRE fits into common definitions of crowdsourcing. Next, the chapter covers requirements engineering and stakeholder analysis and establishes that the research in this dissertation primarily concerns narrow stakeholder sets, or stakeholders with a direct interest in the system. After covering crowdsourcing and the requirements process separately, the chapter considers how CrowdRE relates to and integrates these two topics. Coverage of the CrowdRE literature will focus on the perceived advantages and disadvantages of CrowdRE, a discussion which will form the basis for the empirical analysis in this dissertation, and the existing gaps in the CrowdRE literature.

After covering CrowdRE, attention will shift to network analysis. The review will start by providing a general overview of the topic, and then move to a discussion of network analysis applications within the requirements engineering domain. The discussion will center on methods for constructing stakeholder networks from requirements artifacts and existing research that addresses the influence of stakeholder network structure on the effectiveness of requirements processes.

\section{Requirements Engineering for OSS}
\label{oss_re}

OSS projects have proliferated in recent years with the advent of online code collaboration platforms such as GitHub, GitLab, and SourceForge. Camara and Fonseca~\cite{camara} identify four categories of OSS projects: community-led, corporate-led, academic-led, and innovation-led. The various categories differ with respect to modularity---the degree to which a developer can work on a discrete component of the system---and conceptualization, or the level of visibility and clarity of the underlying source code~\cite{camara}. Each OSS category has distinct characteristics, resulting in different patterns of crowd interaction. This study considers community-led projects, which have broad adoption and depend on a network of volunteer developers to maintain the code base, fix bugs, and implement new features~\cite{camara}. 

The requirements process for community-led OSS projects differs from proprietary software, and even from more strongly organized corporate-led OSS efforts~\cite{scacchi}. Scacchi~\cite{scacchi} describes the requirements process for community-led OSS projects as less formal, with requirements often arising from discussion forums rather than through rigorous stakeholder analysis. This process works for OSS because most OSS users are developers. As a result, they have an unusually strong understanding of the underlying technical details of the OSS project and can formulate reasonably high quality requirements without needing a product manager as an intermediary~\cite{scacchi}. Empirical research by Kuriakose and Parsons~\cite{kuriakose} and Paech and Reuschenbach~\cite{paech} confirm this understanding of the requirements process for OSS.

In addition to requirements engineering, OSS measures of effectiveness have received considerable attention in the literature. Stewart and Gosain~\cite{stewart}, for example, differentiate between input and output measures of effectiveness. Input measures of effectiveness include the attraction and retention of developers while output measures of effectiveness consider factors such as task completion rates and software quality~\cite{stewart}. Crowston and Howison~\cite{crowston} and Ghapanchi, et al. \cite{ghapanchi} develop a similar framework, citing code quality, user satisfaction, developer satisfaction, and contributor turnover rate as key indicators of success for OSS projects. With regard to the requirements process specifically, Ma~\cite{ma} highlights requirement volume and the level of agreement between users on the importance of requirements as key indicators of project success. This study uses these measures to develop an understanding of how stakeholder network structure and crowd participation in the requirements process impacts OSS project success.

\section{Crowdsourcing}

This section considers the origin of crowdsourcing as a concept outside systems engineering. First, the section surveys several definitions of crowdsourcing, focusing first on classic discussions by Surowiecki, Howe, and Brabham before turning to more recent academic literature. Next, the section compares and contrasts different categories of crowdsourcing activity. An analysis of the literature establishes that CrowdRE falls squarely within mainstream conceptions of crowdsourcing.   

\subsection{Definitions of Crowdsourcing}

Surowiecki~\cite{surowiecki} initially conceived of crowdsourcing in his 2004 book, \emph{The Wisdom of Crowds}. According to Surowiecki, the average decision making capacity of a crowd of non-experts exceeds that of an individual expert, even in fields that normally require advanced training, citing markets and betting spreads as examples. In both circumstances, the aggregate decision of a large group frequently converges to the correct solution, even when experts have difficult making accurate predictions. Surowiecki expected that the appeal of crowd wisdom would grow with the emergence of web-based tools capable of organizing and coordinating crowds.

Building on the work of Surowiecki, Howe~\cite{howe, howe2} brought crowdsourcing into the popular and scientific lexicon with a 2006 article and subsequent book that explains how organizations can achieve superior results at lower cost by outsourcing tasks to the crowd. For Howe, crowdsourcing represents the process of outsourcing tasks to a network of contributors through an open call for participation. As such, Howe conceives of crowdsourcing as a means to harness the wisdom of crowds for the benefit of an organization. 

In agreement with Surowiecki's observation about web-based tools, Howe~\cite{howe} attributes the rise of crowdsourcing in the early 2000s to the development of technology that eased the burden of coordinating and incentivizing disparate contributors across multiple locations. According to Howe, crowdsourcing can revolutionize both skilled and unskilled labor. He claims, for example, that corporate research and development at blue-chip companies such as Eli Lilly, Proctor and Gamble, and DuPont have benefited from web-based platforms such as InnoCentive, which awards monetary prizes to participants who produce innovative solutions to the problems organizations post on the site. Meanwhile, new tools, such as Amazon Mechanical Turk, make it easier for companies to offload simple, time-consuming tasks, such as labeling images for machine learning applications.

Whereas Howe and Surowiecki wrote for broad audiences, Brabham~\cite{brabham, brabham2} explores crowdsourcing from a scholarly perspective in his 2008 publication on the topic. Brabham distills the concept of crowdsourcing into four essential elements: 
\begin{enumerate}
    \item An organization with a task to perform
    \item A crowd of participants willing to perform the task
    \item An online tool to facilitate the performance of the task
    \item Mutual benefits for the crowd and the organization
\end{enumerate}
Notably, Brabham~\cite{brabham} excludes OSS projects and commons-based peer production platforms such as Wikipedia from his crowdsourcing definition. He does not categorize these collaboration patterns as crowdsourcing due to the lack of a central organization that coordinates and benefits from the activity of the crowd. As the abundant literature on crowdsourcing for OSS attests~\cite{glinz}, however, this view does not go unchallenged. Indeed, the picture Brabham~\cite{brabham} paints of OSS---one of hobbyist hackers making spontaneous, uncoordinated contributions---starkly underestimates the degree of organization required to manage a major OSS project. Likewise, in the case of Wikipedia, the Wikimedia Foundation plays a greater role in setting and enforcing community standards than Brabham supposes and the organization ultimately benefits from crowd contributions.

Vukovic~\cite{vukovic} provides an alternative definition in her 2009 article discussing crowdsourcing for enterprises. For Vukovic, all crowdsourcing processes have four steps: registration and specification, the initial crowdsourcing request, execution of the task, and the completion of the task. Registration and specification includes activities such as defining the crowdsourcing request and finding a platform to host the request. During the initial crowdsourcing request phase, the requesting organization publicizes the request on an appropriate platform and recruits participants. Execution occurs when the participants begin work on the task. Finally, participants complete the task and the requesting organization compiles the results. Vukovic agrees with Brabham about the necessity of a web-based platform and categorizes crowdsourcing under the broader umbrella of Web 2.0. Unlike Brabham, Vukovic does not exclude OSS from her crowdsourcing definition.

Estelles-Arolas and Gonzalez-Ladron-de-Guevara~\cite{estelles} attempt to reconcile the proliferation of crowdsourcing definitions from Brabham, Vukovic, and others. They arrived a definition that consists of several elements. First, they concur with Brabham and Vukovic, concluding that crowdsourcing necessitates an online medium for coordination. Second, the organization must have a task to complete, for which they issue an open call for volunteers. Third, the task must have participants who perform the task and receive some benefit in return. Benefits can include payments, personal satisfaction, and improved social standing. Finally, the organization must utilize contributions from the crowd for its benefit. Overall, despite the influence of several dozen independent definitions, their conception of crowdsourcing remains remarkably well aligned with the original definitions from Howe, Brabham, and Vukovic, revealing a strong consensus about what constitutes crowdsourcing.

While most of the early writing views crowdsourcing in strictly optimistic terms, Schenk and Guittard~\cite{schenk} take a more measured approach and highlight both the positive and negative aspects of crowdsourcing. For Schenk and Guittard, the benefits of crowdsourcing include lowering costs, increasing the diversity of potential solutions, positive network effects, and stronger motivation due to the presence a robust community. As drawbacks, they highlight the transaction costs involved in the coordination of crowd members, uncertainty about the level of contribution from the crowd, and inability to effectively define tasks for the crowd. Collectively, these advantages and disadvantages outline the major challenge inherent in crowdsourcing: although crowdsourcing brings more manpower and diversity of thought to bear on a problem, managing the process imposes considerable costs on an organization. To make informed choices, engineers need an empirical basis for navigating this trade-off.

\subsection{Crowdsourcing in a Systems Context}

Doan, Ramakrishnan, and Halevy~\cite{doan} make the first attempt to define crowdsourcing within a systems context. Specifically, they define a crowdsourcing system as a system that uses inputs from the crowd to solve a problem that the system owners have defined. Crucially, this definition does not require crowd members to act intentionally on behalf of the system. With this in mind, Doan, Ramakrishnan, and Halevy differentiate between explicit and implicit crowd participation.

Doan, Ramakrishnan, and Halevy~\cite{doan} describe four major challenges inherent to crowdsourcing: recruiting crowd-participants, determining how crowd members should participate, managing their contributions, and forestalling abuse. While managing these challenges, organizations face a fundamental trade-off between the volume and quality of crowd contributions. Doan, Ramakrishnan, and Halevy agree with Schenk and Guittard that digital applications provide a more favorable environment for crowdsourcing and focus their study on web-based crowdsourcing tool. Unlike Schenk and Guittard, however, they do not preclude the potential for crowdsourcing to function effectively in the physical domain. Indeed, the emergence in recent years of 3D printing and open source production blueprints has made crowdsourced manufacturing processes more viable than ever~\cite{kostakis}.

\subsection{Crowdsourcing Taxonomies}

Vukovic~\cite{vukovic} categorizes crowdsourcing efforts based on the crowdsourced function and crowdsourcing mode. The crowdsourced function represents the stage of the system lifecycle that benefits from the crowdsourcing project. Vukovic highlights four possibilities---design, development and testing, marketing and sales, and support---although approaching the problem from a systems engineering lens would certainly yield many more. The crowdsourcing mode relates to the platform for managing the request and the means for distributing benefits to the crowd. For example, paying a fixed price for data labeling would be a different mode of crowdsourcing than offering a prize for inventing a new solution to a problem. Ultimately, Vukovic identifies marketplaces and competitions as the primary modes of crowdsourcing. Vukovic sorts a variety of contemporary crowdsourcing examples based on their function and mode.

Schenk and Guittard~\cite{schenk2} formulate an alternative taxonomy with three components:
\begin{itemize}
    \item Members of the crowd, who provide input
    \item A company or organization that benefits from crowdsourcing
    \item A mechanism for coordination between the organization and the crowd
\end{itemize}
Due to the higher overhead costs involved in distributing and coordinating activities for physical production, Schenk and Guittard~\cite{schenk2} consider crowdsourcing most appropriate for virtual, knowledge-based tasks. Schenk and Guittard make the distinction between two types of crowdsourcing tasks: integrative and selective. Integrative crowdsourcing tasks generally involve distributed data collection, whereas selective crowdsourcing harnesses the creative power of individuals outside the organization. Under this definition, image labeling using Mechanical Turk is an integrative task, whereas InnoCentive challenges are selective tasks.

Using the nature of collaboration and the means for addressing the four challenges as a baseline, Doan, Ramakrishnan, and Halevy~\cite{doan} develop a more systems-focused crowdsourcing taxonomy. According to Doan, Ramakrishnan, and Halevy, crowdsourcing systems with explicit collaboration involve five categories of user tasks: evaluation, sharing, networking, task execution, and the creation of system artifacts. Typically, crowdsourcing systems with explicit participation maintain a dedicated platform for collaboration and require organizations to recruit crowd participants. By contrast, implicit crowdsourcing systems benefit from participants performing activities unrelated to the core purpose of the system. For example, consider a system that uses bets in a prediction market to improve a forecasting model. Crowd participation improves the prediction model, even though winning bets remains the sole focus of the users themselves. Implicit crowdsourcing systems do not always need to recruit participants, and may leverage platforms not intentionally designed for crowdsourcing. 

\subsection{Application to Current Research}

The literature on crowdsourcing begs the question: how well does requirements crowdsourcing for OSS project meet the standard definitions? The earlier, narrower definitions from Howe~\cite{howe, howe2}, Brabham~\cite{brabham, brabham2}, and Vukovic~\cite{vukovic} largely exclude this type of requirement solicitation due to, in most cases, the lack of an organized call for participation from the crowd and the lack of explicit rewards for participants. The concept of implicit crowdsourcing~\cite{doan}, however, can accommodate the elicitation of requirements from the crowd without an explicit reward. As in the example of betting markets, participants in the requirements process for crowdsourcing contribute primarily for their own benefits---they want the organization to implement features that would enable them to utilize the system more effectively. Participation in a community itself also provides direct intangible benefits to crowd members, such as a sense of belonging to a community~\cite{levy}. Nevertheless, the organization benefits as a result of the largely self-interested activity of the crowd. As a result, the style of crowd collaboration that forms that basis of this research falls squarely within established conceptions of implicit crowdsourcing. This aligns with the prevailing view in the CrowdRE literature. Glinz~\cite{glinz}, for example, categorizes the elicitation of OSS requirements from the crowd as a crowdsourcing task.

\section{Stakeholder and Requirement Analysis}

From a systems engineering perspective, the research in this dissertation covers two major components of the system decision making process: stakeholder analysis and requirements generation. This sections considers how the research in this dissertation fits into the appropriate International Council on Systems Engineering (INCOSE) technical processes, with a specific focus on the stakeholder needs and requirements definition processes. The stakeholder needs discussion considers different stakeholder classifications and their ramifications for the construction of stakeholder networks. In this research, the stakeholder networks consist of core stakeholders, or those who engage directly with the system. The requirements engineering section starts with a broad discussion of the requirements process, and then dives into a more tailored discussion of CrowdRE. The final section covers the benefits and challenges of CrowdRE. The trade-offs inherent to CrowdRE serve as the foundation for the empirical analysis in this dissertation.

\subsection{Systems Engineering Technical Processes}

This research addresses the quality of system requirements sourced from the crowd. Placing this research in context necessitates a baseline understanding of how requirements analysis fits into the broader set of systems engineering technical processes. INCOSE provides the most commonly accepted definition of these technical processes. According to INCOSE~\cite{incose}, requirements gathering occurs after the "mission analysis process", which "identifies stakeholders and uses cases", and before the "architecture definition process", which charts out the "components of the technical solution". INCOSE~\cite{incose} splits requirements gathering into two distinct processes: the "stakeholder needs and requirements definition process" and the "system requirements definition process". The first focuses on identifying the business needs of stakeholders, whereas the second outlines the technical capabilities that must exist to meet those needs~\cite{incose}. This research primarily addresses the stakeholder needs and requirements definition process because crowd members typically express functional needs. Although some highly technical crowd members have enough familiarity with an OSS system to generate fully formed system requirements, the refinement of crowdsourced requirements normally occurs after the elicitation of those requirements and before implementation.

\subsection{Stakeholders}

This section outlines two major approaches to stakeholder analysis. Mitchell, Angle, and Wood~\cite{mitchell} offer a classic formulation that categorizes stakeholders based on legitimacy, power, and urgency. Newcombe~\cite{newcombe} provides an alternative approach and distinguishes between core and peripheral stakeholders. The research in this dissertation considers both of these ontologies to determine which stakeholders to include in the stakeholder network and how that situates the analysis within the broader stakeholder ecosystem.

\subsubsection{Mitchell, Angle, and Wood}

Mitchell, Angle, and Wood~\cite{mitchell} provide an authoritative definition of what constitutes a stakeholder, along with a taxonomy that details different types of stakeholders. According to Mitchell, Angle, and Wood, the set stakeholders encompasses any entity that can impact the system and any entity that the system can impact. Stakeholders include both groups and individuals. 

Mitchell, Angle, and Wood~\cite{mitchell} categorize stakeholders based on three characteristics: power, legitimacy, and urgency. Power refers to the ability of a stakeholder to influence other actors within the system. Legitimacy involves the perception of whether or not stakeholders behave appropriately within the system, given the assumed role of the stakeholder. Urgency refers to ability of a stakeholder to command immediate attention. Based on these characteristics, Mitchell, Angle, and Wood derive numerous categories of stakeholders:

\begin{itemize}
    \item \emph{Definitive Stakeholders} have power, urgency, and legitimacy.
    \item \emph{Dominant Stakeholders} have power and legitimacy, but no urgency.
    \item \emph{Dependent Stakeholders} have urgency and legitimacy, but no power.
    \item \emph{Dangerous Stakeholders} have power and urgency, but no legitimacy.
    \item \emph{Dormant Stakeholders} have power, but not legitimacy or urgency.
    \item \emph{Discretionary Stakeholders} have legitimacy, but no power or urgency.
    \item \emph{Demanding Stakeholders} have urgency, but no power or legitimacy.
\end{itemize}

Mitchell, Angle, and Wood~\cite{mitchell} also differentiate between broad and narrow stakeholder sets. The broader stakeholder sets includes any group or individual with a stake in the system, however minor. By contrast, narrow stakeholder sets constitutes only stakeholders with a direct interest in and the ability to influence the system.

\subsubsection{Newcombe}

Working within the context of construction project management, Newcombe~\cite{newcombe} develops a separate set of procedures for identifying stakeholders and judging their level of predictability, interest, and power over a project, which Nebcombe refers to as \emph{stakeholder mapping}. As part of the stakeholder mapping process, Newcombe encourages systems engineers to assess stakeholders using three criteria:

\begin{itemize}
    \item The likelihood of a stakeholder asserting influence over the project.
    \item The ability for a stakeholder to influence a project.
    \item The impact of stakeholder expectations on project management outcomes.
\end{itemize}

The Newcombe framework largely echoes the stakeholder categories that Mitchell, Angle, and Wood identified. Newcombe~\cite{newcombe} also differentiates between stakeholders within a project---those who directly contribute to or benefit from it---and those outside a project, who only have an indirect stake. The distinction between core and peripheral stakeholders closely corresponds to the definition of broad and narrow stakeholders in Mitchell, Angle, and Wood. The broad agreement between Newcombe~\cite{newcombe} and Mitchell, Angle, and Wood~\cite{mitchell} reflects a reasonably strong consensus within the literature about how to identify and categorize stakeholders.

\subsubsection{Implications for this Research}

This research primarily considers narrow stakeholder sets; namely, individuals who actively contribute to or provide feedback for OSS projects. The stakeholders in this research span several of the categories from Mitchell, Angle, and Wood~\cite{mitchell}. Definitive stakeholders consist of active core developers on an OSS project, who prioritize and implement requirements. If a developer ceases contributing to a project, thus losing urgency, they can slip into the dominant or dormant stakeholder categories. Demanding and dependent stakeholders include users of an open source tool who submit requirements, but do not themselves contribute to the code base. Within the context of OSS, dangerous and discretionary stakeholders do not exist in any meaningful sense because, in most cases, power implies legitimacy and legitimacy implies power. Users without sufficient permissions (legitimacy) on an OSS project cannot merge code without review (power), and the possession of those permissions (legitimacy) means a developer has the ability to change the state of the system (power). The categories of stakeholders outlined in this section will frame the discussion of stakeholders throughout this body of research.

\subsection{Requirements}

The INCOSE handbook~\cite{incose} outlines the major activities in the requirements process, which include defining, analyzing, and managing system requirements. Finkelstein~\cite{finkelstein} expands upon these, noting several activities within each stage of the process. Defining requirements involves initial groundwork aimed at bounding the scope of the problem and encouraging participation in the requirements process. Requirements analysis entails several components, including feasibility and risk assessments, value modeling to determine priorities, and the development plans to validate and verify requirements. The management of system requirements involves establishing metrics to measure the effectiveness of the requirements, estimating task completion time, verifying and validating results, and maintaining knowledge management systems to trace the status of and dependencies between requirements. 

The research in this dissertation will not address every step in the requirements process. Specifically, this research focuses on the requirements process at an aggregate level and, as a result, primarily considers measures of effectiveness for the requirements process as a whole. Topics such as feasibility analysis, requirement verification and validation, and time-to-completion estimates for individual tasks do not figure heavily into the discussion.

\subsection{Crowdsourcing for Software Engineering}
\label{crowd_in_se}

With the emergence of adequate tools, crowdsourcing has grown more prevalent, particularly in the field of software engineering. According to Mao, Capra, Harman, and Jia~\cite{mao}, published research on crowdsourcing within the software engineering domain grew from only one paper in 2008 to over 200 in 2015. When Mao and his colleagues reviewed the crowdsourcing landscape within software engineering, however, they found a heavy focus on implementation, with a comparative lack of emphasis on the requirements process. Specifically, they found that most crowdsourcing focused on three entities: a \emph{requester} who needs a task performed, a \emph{platform} for facilitating completion of the task, and a \emph{worker} for performing a task. This paradigm presupposes the existence of a complete design, which workers simply need to implement. Indeed, Mao and his colleagues found that only 6\% of research focused on applications related to requirements acquisition.

Writing contemporaneously with Mao, LaToza et al.~\cite{latoza2} claim that software engineering teams can derive a host of benefits from crowdsourcing, to include shorter time to market, reduced development costs, increased developer productivity, and availability of technical expertise not available internally. Systems engineers can tap into these benefits through the use of techniques such as peer participation, competitions, and micro-tasking. As part of the micro-tasking framework, systems engineers decompose requirements into independent, atomic tasks, which they then outsource to crowds of developers~\cite{latoza}. Stol, LaToza, and Bird~\cite{stol} emphasize four categories of crowdsourcing tasks: rating, creation, processing, and problem solving. Consistent with observations from Mao et al., treatment of the topic within LaToza's line of research focuses primarily on implementation and lacks a systematic treatment of requirements or stakeholder analysis. Although Stol, LaToza, and Bird~\cite{stol} argue that crowdsourcing facilitates quicker identification of bugs and more productive feedback and briefly note requirements elicitation as a potential application, they do not discuss the implication of these outcomes for stakeholder analysis more broadly.

\section{Crowd-Based Requirements Processes}

The literature on crowd-based requirements processes (CrowdRE) contains three important threads: the underlying need for crowdsourcing, the advantages and disadvantages of crowdsourcing, and methods for motivating crowd participation. CrowdRE emerged out of a need to receive and implement feedback about unanticipated system uses cases~\cite{groen}. Crowdsourcing benefits systems engineers by facilitating quicker stakeholder feedback and developing a community invested in the success of the system. Disadvantages include increased project management overhead and the need to incentivize crowd members to remain engaged. Researchers have developed several approaches to maintaining crowd engagement, including gamification and encouraging the formation of cohesive micro crowds. The advantages and disadvantages of CrowdRE that the literature articulates form the basis for the empirical analysis in this research.

\subsection{The Need for CrowdRE}

As early as 2002, Finkelstein~\cite{finkelstein} anticipated the importance of crowdsourcing for requirements engineering, despite, at the time, a lack of platforms to facilitate distributed, asynchronous collaboration on projects. In particular, Finkelstein notes user engagement for information gathering and task analysis as domains ripe for future requirements engineering research. Finkelstein highlights information gathering as the most difficult task within the requirements engineering process and suggests traditional requirements engineering techniques do not take the needs and work patterns of ordinary users sufficiently into account. Finkelstein also underscores the importance of discovering how and for what purpose users interact with a system as part of the task analysis process.

\subsection{Advantages and Disadvantages}
 
In contrast to the discussion of crowdsourcing within software engineering, the requirements engineering literature features a lively discussion of how crowdsourcing intersects with stakeholder analysis. Although the literature has largely focused on implementation, Glinz \cite{glinz} suggests that project managers can also crowdsource system requirements. Gerth, Burnap, and Papalambros \cite{gerth} concur, but argue that while crowdsourcing can improve the quality of requirements under the right circumstances, crowdsourcing only produces these benefits if systems engineers institute well-governed processes for managing the crowd. Advantages of CrowdRE include the ability to solicit a broader diversity of designs and requirements, the quicker collection of feedback, and an increased feeling of ownership within the user base, which helps fuel further product adoption. Challenges associated with CrowdRE include the need to sift through a large number of contributions of varying quality, lack of sufficient expertise within the crowd, and concerns about how well the crowd represents the needs of stakeholders more broadly. Effectively balancing the advantages and disadvantages proves difficult because different use cases and crowd sizes demand different approaches.

The requirements engineering community has spent considerable effort to develop tools and processes for effectively managing crowds. In addition to proposing solutions, this literature contains ample discussion about the drawbacks of crowdsourcing. Finkelstein et al.~\cite{stakenet, stakerare, lim} developed software packages such as StakeRare and StakeNet to address concerns about the ability of systems engineers to prioritize requirements effectively when faced with an overwhelming volume of feedback from the crowd. Their network-based approach uses centrality measures to identify the most influential stakeholders in the system and allows project managers to prioritize requirements accordingly. Finkelstein et al.~\cite{stakenet, stakerare, lim} claim that crowd-centric techniques enable project managers to discover use cases and stakeholders that they failed to identify during the system design phase. By engaging with crowds early in the utilization process, project managers can iterate on the design and avoid investment in solutions that do not truly address stakeholder needs.

Groen, Dalpiaz, and Ali~\cite{groen} introduced the notion of CrowdRE to more strongly emphasize the role of the user in the requirements process. They differentiate CrowdRE from several related approaches, including customer-specific requirements engineering, market-driven requirements engineering, and crowdsourcing more broadly. In their view, CrowdRE supersedes customer-specific and market-driven requirements engineering by serving the same functional purpose, while at the same time expanding the scope of the requirements process to consider the views of a more diverse set of stakeholder. CrowdRE distinguishes itself from traditional crowdsourcing because it values the input and opinions of crowd members. By contrast, traditional crowdsourcing, as noted in Section \ref{crowd_in_se}, focuses on the crowd as a resource for accomplishing a task. According to Groen~\cite{groen}, crowd members within a requirements generation context, while still performing a task, feel more ownership due to the critical role they play in influence the design and direction of the system.

While optimistic about the potential of CrowdRE, Groen et al.~\cite{groen} also highlight potential drawbacks, including a tendency for project managers to overlook the needs of critical stakeholders who engage less actively with the rest of the network. Snijders et al.~\cite{snijders, snijders2} cite a limited view of stakeholders as the primary reason for overlooking stakeholder needs. Other concerns include keeping crowd members motivated, eliciting honest feedback, and analyzing a large volume of requirement. Groen, Dalpiaz, Ali,  et al.~\cite{groen} insist that crowd members need an incentive to produce high-quality requirements. Snijders and Dalpiaz~\cite{snijders, snijders2} develop a solution based on gamification, in which systems engineers offer explicit rewards to induce crowd members to remain engaged. As the results chapter will show, gamification and other methods for motivating crowd participation are especially important when the level of crowd engagement for a project is low.

\subsection{Incentives and Crowd Motivation}

Hosseini, Phalp, Taylor, and Ali~\cite{hosseini} assess that properly incentivizing crowd members produces a host of benefits, including an improved ability to adapt requirements to changing conditions, empirical validation of assumptions about the system, and the identification of new stakeholders and use cases. These advantages accrue most readily to systems that contend with more uncertainty. They also outline common features of the crowd within the context of requirements engineering. They define the crowd based on diversity with respect to location, age, gender, and expertise, the level of familiarity of the crowd with the system, the size of the crowd, and suitability of the crowd in terms of expertise and motivation. Hosseini et al.~\cite{hosseini} identify the following characteristics in different crowd types:

\begin{itemize}
    \item Large crowds increase the volume of information, but make management more difficult
    \item Anonymous participation leads to more honest feedback, but makes it more difficult for project managers to assess credibility
    \item Increased diversity reduces consensus, but allow systems engineers to consider more viewpoints
    \item More motivated and incentivized crowds participate more actively and produce better requirements
\end{itemize}

Levy, Hadar, and Te'eni~\cite{levy} agree that crowd motivation presents a risk. They propose an alternative solution to mitigate it based on what they call micro-crowds. In the early stages of a project, their strategy calls for the development of smaller, more cohesive micro-crowds based on the theory that crowd members have a higher probability of remaining engaged if they form strong bonds with fellow participants. As a project expands, project managers work to connect independent micro-crowds and establish a broader community. The micro-crowds approach empowers project managers to encourage connections that benefit the system as a whole, rather than merely relying on the organic behavior of the crowd. By starting small and ensuring contributors remain actively engaged, the micro-crowd approach aims to increase user satisfaction and produce higher quality requirements for the system. The results of this research suggest that micro-crowds are not only important for motivating the crowd, but also promote the formation of the types of stakeholder networks that correlate with strong project management outcomes.

More recent research by Khan, et al.~\cite{khan} includes definitions for key constructs in CrowdRE. Three concepts form the core units of analysis for CrowdRE: the crowd, the task, and the mechanism. Khan, et al.~\cite{khan} define the crowd as the set of entities engaged in the requirements process and categorize crowds according to their scale, role, and level of expertise. The task describes the activity within the requirements process that the crowd performs, and the mechanism facilitates task completion. Examples of tasks that crowd members perform include requirements generation, modeling, validation, analysis, prioritization, and evolution~\cite{khan}. Wang, et al.~\cite{wang} arrive at the same set of requirements activities and further differentiate between explicit user feedback, in which crowd members actively contribute, and implicit user feedback based on observing user activity. This research considers explicit feedback expert level crowds of developers engaged in requirements generation. The nature of the crowds under consideration limit how well the results in this paper generalize. Specifically, the results apply to stable OSS projects with strong user adoption, but do not generalize to proprietary software, hardware, or OSS projects that are earlier in their system life cycle.

\section{Network Analysis}

At its most fundamental level, network analysis contends with objects called graphs, which consist of nodes and edges~\cite{diestel}. Researchers often categorizes graphs based on two characteristics: edge weights and directionality. In unidirected graphs, edges do not have a source or a target---they represent two way relationships. By contrast, directed edges capture the concept of a flow. Weighted graphs apply a numerical value to each edge, which typically represents distance, cost, or another analogous value. Unweighted graphs treat all edges equally. 

This section will consider two aspects of network theory and how they relate to the current research: applications of network theory and network topology. The applications discussion concerns how common use cases for network analysis relate to system engineering. In particular, stakeholder networks borrow heavily from the literature of social networks more broadly. The topology section discusses three categories of networks---random graphs, small-world networks, and scale-free networks---and discusses implications for the study of stakeholder networks.

\subsection{Applications of Network Theory}

Although this dissertation will focus on systems engineering applications, a general overview of network theory can help put these applications in context. Network theory has played an increasingly prominent role in several academic fields since the turn of the 21st century. In the literature, graphs can represent a host of physical concepts, including telecommunications networks and the flow oil through as set of interconnected pipelines~\cite{barbasi}. The study of social networks, however, informs the use of network analysis for requirements engineering most directly.

The concept of social network analysis dates back to Jacob Moreno~\cite{moreno}, who constructing sociograms---graphs that depict people as nodes and relations between people as edges. In the years since, researchers have applied social network analysis to a variety of fields, including epidemiology to study the spread of disease~\cite{moreno}, economics to study the geographical clustering of industries~\cite{moretti}, and international relations to study the propagation of diplomatic influence~\cite{slaughter}. 

Although the applications span diverse settings, they share several important characteristics. First, they all note the importance of network hubs, or nodes that have disproportionately high connectivity. Within social networks~\cite{barbasi}, these nodes serve as conduits for information flow. Second, they recognize the importance of network structure to the functioning of networks. Power brokers, for instance, have more control in concentrated networks than in disperse networks~\cite{slaughter}. Finally, they recognize social networks as dynamic entities that develop over time. This gives humans---including systems engineers---the ability to influence social networks through well-crafted policies.

\subsection{Common Network Topologies}

Network theorists have proposed several common topologies relevant to social networks. These include random graphs, small-world networks, and scale free networks. Random graphs~\cite{erdos} connect nodes with equal probability, small-world networks~\cite{watts} include cohesive and interconnected communities, and scale-free~\cite{barbasi} networks preferentially add connections to the highest degree nodes. Small-world networks and scale-free networks influence the study of stakeholder networks most directly, and provide a basis for measuring stakeholder network structure.

\subsubsection{Random Graphs}

In a random graph, each node has an equal probability of connecting with any other node. Erdos and Renyi~\cite{erdos} showed that constructing graphs according to a random process results in a Poisson distribution for the edge degree of nodes. Although interesting from a theoretical perspective, empirical analysis shows that random graphs do not often appear in real world applications~\cite{barbasi}. As a result, these graphs do not have major practical implications for systems engineers.

\subsubsection{Small World Networks}

Watts and Strogratz~\cite{watts} propose another topology with more direct relevance for systems engineering: the small world network. Features of small world networks include low average path length between nodes and a high degree of clustering. This implies the formation of distinct communities with the network, with small number of nodes acting as connectors between communities. In the real world, small world networks appear in a variety of settings, including in social networks, coauthor networks for academic papers, and actor collaboration networks for movies, the last of which has given rise to the "Bacon number", the length of the minimum path between a given actor and Kevin Bacon. Somewhat ironically, Paul Erdos, who formulated the random graph, serves as the equivalent point of reference for academic coauthor networks~\cite{barbasi}.

Unlike random graphs, small world graphs have a high likelihood of appearing in a requirements engineering context due to the social nature of requirements collaboration. The following narrative demonstrates how such a network could develop. A project consists of several core contributors, each of whom specializes in a particular component of the system. Groups of users develop within certain system use cases, which to varying degrees relate to specific system components. A network develops in which the maintainer of a system component interacts frequently with users whose interests intersect with that component, and interaction between project contributors connect the communities that forms as a result. While this dynamic requires empirical evaluation, it stands to reason that small-world networks would naturally arise from requirements collaboration.

\subsubsection{Scale-Free Networks}

Barbasi and Albert~\cite{barbasi2} propose scale-free networks, in which the degree distribution of the nodes follows a power law distribution. In these graphs, new connections show \emph{preferential attachment} to nodes that already have a high edge degree. Preferential attachment occurs in situations with increasing returns to scale. For instance, in the case of web traffic, more users may visit an authoritative site. In turn, increased traffic makes the site seem more authoritative, driving additional traffic.

Potential analogs exists within the field of requirements engineering. For instance, an expert contributor may collaborate on many requirements due to specialized knowledge, in turn driving the development of more specialized knowledge and more preferential collaboration. Requirements collaboration networks, however, have built-in limits due to the need for stakeholder involvement. No matter how much expertise an contributor has, he still has a limited amount of time each day during which to contribute. Requirements collaboration networks, therefore, appear unlikely to achieve the levels of concentration necessary to produce a scale-free distribution of edge degrees.

\subsubsection{Applications to Research}

This dissertation focuses on the practical impacts of network structure for systems engineers. As a result, theoretical discussions about network topology have only secondary importance. There literature on network topology, however, does impact this research in two important ways. First, common network topologies point to the use of specific metrics for measuring the structure of stakeholder networks. Small-world networks suggest using measures such as the distance between nodes and the degree of clustering in the network. Scale-free networks place importance on measuring the level of concentration within the degree distribution of the nodes. Second, determining the likely structure of stakeholder networks within a requirements engineering context could have implications for study dynamic stakeholder networks. For instance, if systems engineers know that most stakeholder networks have small-world structures, they could use that knowledge to develop simulations.

\section{Collaborative Requirements Elicitation and Stakeholder Networks in Systems Engineering}
\label{network_re}

The previous section covered network analysis from a theoretical perspective. This section will explore applications of network theory within systems engineering, with a focus on stakeholder analysis. Collaborative requirements elicitation \cite{stakerare, stakenet, stakesource, mobasher} uses social networks to identify new stakeholders, organize the input of existing stakeholders, and prioritize requirements based on common needs. In addition to providing a basis for the use of social networks for requirements engineering, the collaborative requirements elicitation literature develops an approach to structuring a large volume of feedback from the crowd. Lim, et al. \cite{stakerare, stakenet, stakesource, lim} suggest using network structure and collaborative filtering to group and prioritize requirements. Mobasher and Cleland-Huang \cite{mobasher} use recommendation systems to accomplish the same end. In addition to drawing inspiration from the use of social networks in the collaborative requirements elicitation literature, this study highlights opportunities to employ these techniques in an OSS setting. Two broad themes emerge within the literature: using network centrality to measure the influence of individual stakeholders and studying the implications of network structure for the effectiveness of the system as a whole. This research falls into the latter category, but borrows important concepts from the former.

\subsection{Network Structure and Influence}

The first major category of stakeholder network analysis within systems engineering concerns the use of centrality measures to identify influential stakeholders. Actor-Network Theory (ANT)~\cite{missonier} posits that network structure has important implications for the power dynamics within a project. The theory implies that systems engineers can influence project management outcomes through interventions that alter the structure of the stakeholder network. From a more practical perspective, Linaker, Regnell, and Damian~\cite{regnell} demonstrate that stakeholders with more centrality within a network have a stronger influence on requirement prioritization. Wood, Sarkani, Mazzuchi, and Everleigh~\cite{wood} develop an approach called stakeholder crosswalks, which builds stakeholder networks from project management artifacts. They use the resulting networks to identify patterns of stakeholder influence. Finally, Damian, Marczak, and Kwan~\cite{damian} proposed requirements-centered social networks and focus on how knowledge of requirements propagates through stakeholder networks in a remote working environment. Lopez-Fernandez, et al. \cite{lopez} apply stakeholder analysis to identify key influencers amongst stakeholders in OSS projects within the Apache ecosystem.

\subsubsection{Actor-Network Theory}

Missonier and Loufrani-Fedida~\cite{missonier} developed an approach based on actor-network theory (ANT), which describes stakeholder networks as dynamic social constructs whose characteristics and effectiveness change over time. Although ANT originally sought to address sociological phenomena~\cite{latour}, Missonier and Loufrani-Fedida~\cite{missonier} note that ANT also serves as a useful framework for discussing the interplay between the social and technical components of project management. Using ANT as a baseline, Missonier and Loufrani-Fedida~\cite{missonier} carry out a case study in which they build stakeholder networks by creating nodes for each stakeholder with an interest in the project and adding edges to represent between stakeholders. The stakeholder networks in the case study use unweighted, undirected edges.

Missonier and Loufrani-Fedida~\cite{missonier} use stakeholder networks primarily to evaluate social and power dynamics as they relate to project management. Although the research in this dissertation focuses on project management outcomes rather than the relative ability of various stakeholders to influence a project, Missionier and Loufrani-Fedida reveal an insight that has important implications for this work. Namely, the structure of stakeholder networks change over time, and the nature of those changes impacts how stakeholders engage with one another and how well they perform their functions within the system. From a theoretical perspective, this means that when systems engineers intervene to encourage or discourage crowdsourcing, the resulting changes in network structure may have measurable effects on project management outcomes. This research tests that hypothesis empirically. 

\subsubsection{Stakeholder Influence Analysis}

Linaker, Regnell, and Damian~\cite{linaker} also consider of how network structure impacts the level of influence a stakeholder has over a project. In contrast to Missonier and Loufrani-Fedida, however, they focus on the prioritization of requirements rather than conflict resolution. To construct stakeholder networks, Linaker, Regnell, and Damian create a node for each stakeholder and add directed edges between any pair of stakeholders who have interacted on a common requirement. Linaker, Regnell, and Damian also weight the edges based on the level of participation for each stakeholder.

Using requirements generation for the Apache Hadoop OSS ecosystem as a case study, Linaker, Regnell, and Damian~\cite{linaker} show that stakeholders who occupy central positions within a stakeholder network more strongly influence the prioritization of requirements~\cite{linaker}. The stakeholder network for the case study only includes large organizational stakeholders, such as corporations, universities, and OSS non-profits. By contrast, this research focuses on networks of individual contributors and seeks to understand project management outcomes for the system as a whole rather than the prioritization of individual requirements. Nevertheless, their work provides an important foundation for the research in this dissertation insofar as it outlines procedures for building stakeholder networks from requirements data. As the methodology chapter will later detail, this research borrows heavily from the approach Linaker, Regnell, and Damian developed for building stakeholder networks.

\subsubsection{Stakeholder Crosswalk}

Wood, Sarkani, Mazzuchi, and Eveleigh~\cite{wood} offer an alternative approach to building stakeholder networks based on a method they call the \emph{stakeholder crosswalk}, which applies specifically to the Department of Defense Architectural Framework (DoDAF)~\cite{dodaf}. Systems engineers can construct a network using the stakeholder crosswalk by applying the following steps to each of the five DoDAF program phases:

\begin{itemize}
    \item Identify stakeholders and appropriate DoDAF models.
    \item Construct an adjacency matrix in which the columns represent stakeholders and the rows represent DoDAF Models.
    \item Convert the adjacency matrix into a stakeholder network in which nodes represent stakeholders and edges represent stakes in a common DoDAF model. 
\end{itemize}

The stakeholder crosswalk shares similarities with Linaker, Regnell, and Damian's (henceforth, LRD) process for building stakeholder networks, but also differs in some important respects. Both processes construct networks that represent stakeholders as nodes and add edges to reflect collaboration. Likewise, both processes use project management artifacts such as comments and e-mails to build stakeholder networks. The LRD and stakeholder crosswalk procedures both include directed edges. However, in the LRD process, the direction represents a chain of replies, whereas in the stakeholder crosswalk they represent lines of influence. Moreover, LRD networks apply edge weights, while the stakeholder crosswalk does not.

\subsubsection{Requirements-Centered Social Networks}

Damian, Marczak, and Kwan~\cite{damian} used network analysis to evaluate the impact of remote work on requirements collaboration within the context of software engineering. Specifically, they consider \emph{requirements centered teams} (RCTs), in which each team member has responsibility for one stage of the three stages of a requirement life cycle: design, implementation, and testing. Damian, Marczak and Kwan introduce the notion of a requirements-centered social network (RCSN), which represents (1) an ideal pattern of interaction between RCT members and (2) the actual pattern of interaction between RCT members. They use this construct to demonstrate that physical distance does not limit the ability of remote software development teams to collaborate effectively on requirements. They also suggest centrality measures as a way to identify the most important conduits for information within the network. 

RCSNs share much in common with stakeholder crosswalk networks. Both make directed connections between stakeholders. In the case of RCSNs, the direction of edges represents awareness whereas in stakeholder crosswalks directions represent lines of influence. Likewise, RCSNs and and stakeholder crosswalk networks do not include edge weights and categorize stakeholders based on their location within the network. The primary point of departure concerns the object of analysis. Wood, et al.~\cite{wood} study the flow of influence, whereas Damian, et al.~\cite{damian} study the flow of awareness. RCSNs differ from LRD networks in the same ways as stakeholder crosswalk networks.

\subsection{Stakeholder Networks and Measures of Effectiveness}

Another important aspect of the literature on stakeholder networks and systems engineering is the relationship between network structure and system performance. Toral, Martinez-Torres, and Barrero~\cite{toral} study the relationship between network concentration affects and the level of engagement in discussion on requirements for OSS packages. They find that networks with higher network concentration have higher levels of engagement. Iyer and Lyytinen~\cite{iyer} explore the relationship between network structure and various measures of effectiveness, including task completion velocity. Like Toral, Martinez-Torres, and Barrero, they conclude that networks with higher levels of concentration produces better outcomes. To date, no studies have considered the joint impact of network structure and requirements crowdsourcing simultaneously. The research in this dissertation addresses that gap.  

\subsubsection{Network Structure and Information Flow}

Toral, Martinez-Torres, and Barrero~\cite{toral} study how network concentration affects the level of engagement on discussion threads related to the open-source Linux operating system. They construct networks by creating a node for each stakeholder and adding a directed edge when one stakeholder responds to another stakeholder. As such, the directed graph captures the flow of information through the network. Toral, Martinez-Torres, and Barrero measure network concentration using the Gini coefficient and hypothesize that higher network concentration results in better information flow. 

The result of the research shows that higher network concentration improves the flow of information within the community because key contributors act as knowledge brokers who help to connect disparate sub-communities. However, they also note that excessive concentration presents barriers to the integration of new community members, and can lead the community to rely on a small number of core contributors. Toral, et al.~\cite{toral} also make a number of keen observations about the structure of stakeholder networks for OSS more broadly. Namely, they note that OSS projects often display a core-periphery dichotomy. To maintain effectiveness, OSS projects need to ensure that core members remain active both as contributors and as brokers for information flow. In research along similar lines, Kluender, et al. \cite{kluender} formulate a mathematical approach based on network centrality to characterize communication distance within networks of collocated software developers, concluding that greater communication distance increases the likelihood of social conflicts.

\subsubsection{Network Structure and Requirement Quality}

Iyer and Lyytinen~\cite{iyer} consider the effect of network concentration on task completion velocity and the diversity of requirements. For their research, they use a data set consisting of 259 projects from SourceForge, an online code repository, that Robinson and Vlas~\cite{robinson} developed to study what Robinson and Vlas call the six Vs for open source requirments: volume, veracity, volatility, vagueness, variance, and velocity. Iyer and Lyytinen~\cite{iyer} construct stakeholder networks by creating nodes for stakeholders and connecting stakeholders who have collaborated with undirected edges. They do not include edge weights in their stakeholder networks. Iyer and Lyytinen~\cite{iyer} measure network concentration based on the variance of degree centrality within the network, a technique they adapted from Robinson and Vlas~\cite{robinson}.

Iyer and Lyytinen~\cite{iyer} show that, in general, OSS projects with more centralized communications structures have higher task completion rates. The advantage of centralizing, however, diminishes as requirement volume increases. For projects that contend with a high volume of requirements, they recommend decentralized project management processes. In addition to developing robust methodologies for constructing stakeholder networks, the studies outlined in this section inform initial expectations about how network structure will affect system performance.

In constrast to Toral, et al., Iyer and Lyytinen~\cite{iyer} do not consider the flow of information through the network. Rather, they  study the implications of stakeholder network structure for system performance as a whole. In that sense, the approach Iyer and Lyytinen present aligns with Missonier and Loufrani-Fedida, whereas the Toral paper has more in common with the Linaker, Wood, and Damian research. A summary of how these papers compare and contrast appears in Table \ref{network_study_comparison}. The comparison shows that research concern network flows typically included directed edges, whereas studies that focus on global characteristics of the network do not.

\begin{table}
\caption{Comparison of Stakeholder Network Studies}
\label{network_study_comparison}
\begin{tabular}{llll}
\hline\noalign{\smallskip}
Study & Flow  & Direction & Weights  \\
\noalign{\smallskip}\hline\noalign{\smallskip}
Missonier, Loufrani-Fedida  & No Flows & Undirected & Unweighted  \\
Linaker, Regnell, Damian & Influence & Directed & Weighted \\
Wood, Sarkani, Mazzuchi, Eveleigh & Influence & Directed & Unweighted \\
Damian, Marczak, Kwan  & Information & Directed & Weighted \\
Toral, Martines-Torres, Barrero & Information & Directed & Weighted \\
Lyytinen, Iyer  & No Flows  & Undirected & Unweighted \\ 
\noalign{\smallskip}\hline
\end{tabular}
\end{table}

\section{Summary}

The literature review covered research spanning three domains: crowdsourcing, the requirements process, and network analysis. This dissertation makes research contributions at the intersection of these three fields, and is the first study to estimate the joint effect of requirements crowdsourcing and stakeholder network structure on measures of effectiveness for OSS projects.

In the crowdsourcing section, a survey of the literature established that soliciting crowd participation in the requirements processes falls within the established parameters of existing crowdsourcing definitions. Within an OSS context, crowds are peculiar insofar as the crowd members have a higher level of technical competency than in a typical crowd because the majority of crowd members are software developers. Moreover, in contrast to traditional crowdsourcing scenarios, crowds in an OSS context are motivated by implicit rather than explicit rewards. Nevertheless, requirements crowdsources for OSS project conforms to prevailing crowdsourcing definitions because there is an organization (the OSS projects) that benefits from outsourcing a task (requirements elicitation) to a crowd whose primary rewards are a more functionality open source dependency and the positive psychological benefits of active community participation.

A review of the literature on the requirements process underscored the value crowdsourcing requirements provides. Namely, traditional requirements processes often fail to anticipate real world use cases~\cite{groen}. Inviting feedback from the crowd enables systems engineers to adjust priorities based on evolving stakeholder needs. Existing frameworks for stakeholder analysis show that CrowdRE mainly concerns core stakeholders---those who directly participate in the generation and scoping of requirements. Although peripheral stakeholders also have an interest in the system, they do not have influence over the requirements process due to their lack of engagement.

CrowdRE combines aspects of the requirements engineering and crowdsourcing literature. The literature~\cite{groen, hosseini} highlights that soliciting crowd input during the requirements produces the following benefits:

\begin{itemize}
    \item More diverse requirements
    \item Higher quality requirements
    \item Quicker collection of feedback
    \item A user base that feels ownership over the success of a system
    \item Surfacing use cases that systems engineers missed during the design phase 
\end{itemize}

With these advantages come several drawbacks, including the following:

\begin{itemize}
    \item A larger volume of requirements to evaluate and prioritize
    \item Inconsistent requirement quality
    \item Insufficient expertise amongst crowd members
    \item Lack of motivated crowd members to provide feedback
    \item Mismatch between the needs of the most active crowd members and the needs of the system more broadly
\end{itemize}

CrowdRE research proposes several solutions to these challenges, including the gamification of participation~\cite{gerth} and the formation of micro-crowds~\cite{levy} to build stronger communities. Specifically, as the results chapter will show, this study validates that CrowdRE techniques for encouraging additional crowd participation are especially beneficial when an OSS project current sources a limited share of requirements from the crowd. As the share of crowdsourced requirements grows, the results show that techniques such as collaborative requirements elicitation~\cite{lim} and recommendation systems~\cite{mobasher} have a greater impact.

The network analysis section opens with a review of existing use cases for network analysis and common network topologies. Since they involve interpersonal collaboration, stakeholder networks share most in common with social networks~\cite{moreno}, albeit often at a smaller scale. Of the three most commonly study network structures small-world networks~\cite{watts} and scale-free networks~\cite{barbasi} provide the most insight for stakeholder networks. Specifically, they suggest the average distance between nodes, the level of clustering in the network, and the level of concentration in the network as key variables of interest in the study of stakeholder networks.

The study of stakeholder networks has recently emerged as a growing area of interest in the systems engineering literature. Actor-Network Theory~\cite{missonier} explores the interplay between network structure and power dynamics within a project management context. Stakeholder crosswalks~\cite{wood} provide a methodology for constructing stakeholder networks from system artifacts. Requirements-centered social networks~\cite{damian} depict the flow of information across stakeholder networks. Additional research from Iyer and Lyytinen~\cite{iyer} and Regnell, et al. \cite{regnell} has focused how network structured affects global measures of system performance, concluding that more concentrated networks produce better outcomes. In the literature on stakeholder networks, researchers tend to use directed graphs to represent information flows and undirected graphs to represent two-way collaboration.

Overall, the literature reflects growing interest in both network analysis and CrowdRE. However, most of the claims within the CrowdRE have not undergone rigorous empirical validation. Moreover, the literature lacks substantial applications of CrowdRE within an OSS context, an area ripe for automated methods due to the availability of public data sets to mine. Finally, no study has assessed how the level of requirements crowdsourcing and the structure of the stakeholder network jointly impact project management outcomes for OSS projects. Indeed, the research in this dissertations constitutes the most expansive analysis of stakeholder networks for OSS projects in the literature to date. This dissertation constructs a data set consisting of network and crowdsourcing variables for 562 OSS projects, more than doubling the size of the largest data set in the existing literature (259)~\cite{iyer}.