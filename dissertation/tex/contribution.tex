\doublespacing
\chapter{Contribution to Research} \label{chap:contribution}

\section{Introduction}

This chapter coves the research contributions from this dissertation and also outlines the research hypotheses. The chapter begins by describing how this research addresses gaps in the existing literature. Namely, this research is the first study to measure the joint effect of requirements crowdsourcing and stakeholder network structure on system performance for OSS projects. Furthermore, this research makes a number of additional practical and scientific contributions, including:

\begin{itemize}
    \item Providing project managers with a basis for deciding when to encourage additional crowd participation
    \item Applying CrowdRE to OSS project management artifacts in a novel way
    \item Demonstrating that crowd engagement faces diminishing marginal returns
\end{itemize}

After discussing research contributions, the chapter fully explains each hypothesis and sub-hypothesis. The hypotheses in this dissertation fall under three main headings:

\begin{itemize}
    \item The effectiveness of crowdsourcing changes as network structure changes.
    \item Crowdsourcing faces decreasing returns to scale.
    \item More concentrated stakeholder network structures encourage stronger project management outcomes.
\end{itemize}

In addition to outlining these hypotheses, the chapter will outline the expected effect of an increase in the share of crowdsourced requirements or a change in the structure of the stakeholder network on each of the six project management measures of effectiveness in the study. These measures of effectiveness are: requirement close-out time, requirement response time,  the average number of comments per requirement, the average number of requirements per crowd member, requirement volume, and the average retention time for a crowd member. The expected behaviors are as follows:

\begin{itemize}
    \item Increasing the share of crowdsourced requirements increases requirement close-out time
    \item Increasing the share of crowdsourced requirements increases requirement response time
    \item Increasing the share of crowdsourced requirements increases the number of comments per requirements
    \item Increasing the share of crowdsourced requirements decreases the average number of requirements per crowd member
    \item Increasing the share of crowdsourced requirements increases requirement volume
    \item Increasing the share of crowdsourced requirements decreases the average retention time for crowd members
\end{itemize}

The chapter will also cover the expected effect of network structure on each measure of effectiveness.

\section{Research Contribution}

This research has both practical and scientific implications. From a practical standpoint, this research suggests that, beyond a certain point, OSS projects benefit more from implementing procedures that channel existing crowd engagement more effectively than from incentivizing additional requirements generation. Within the framework of CrowdRE, this means emphasizing methods that incentivize crowd engagement when the proportion of crowdsourced requirements is low and automated requirements triage methods~\cite{stakesource, mobasher} as the share of crowdsourced requirements rises. Examples of methods for incentivizing additional crowd engagement include gamification~\cite{snijders, snijders2} and micro-crowds~\cite{levy}. Furthermore, OSS projects could benefit from assigning a team member to focus exclusively on engaging with the crowd and managing requirements, enabling the rest of the team to concentrate on implementation.

From a scientific perspective, this paper makes several contributions. First, applying CrowdRE to requirements generation within an OSS context is novel~\cite{glinz}. The application of CrowdRE in this research is the mining of user feedback to generate stakeholder networks for the OSS project. In addition to applying CrowdRE, this research suggests when and how project managers derive the most benefit from CrowdRE. The evidence suggest project managers should employ CrowdRE techniques that encourage additional crowd participation when the share of crowdsourced requirements is low, and switch to methods that make requirements prioritization and triage more efficient as the share of crowdsourced requirements rises. As a consequence of using CrowdRE as part of the research methodology, this study is both an application and a study of CrowdRE.

Second, this research shows that requirements crowdsourcing faces diminishing marginal returns. That is, beyond a certain level of crowd engagement, the overhead of managing additional requirements limits the degree to which an OSS project benefits from requirements crowdsourcing. This empirical observation is new to the literature. Armed with this understanding of crowd behavior, OSS project managers can design interventions to encourage the appropriate level of crowd engagement for their project. The results of the statistical analysis suggest that diminishing marginal returns set in when the share of crowdsourced requirements exceeds approximately 70\%. 

Third, this research adds to the existing discussion~\cite{groen, levy, hosseini, snijders, snijders2} on crowd motivation. Namely, it suggests that project managers should not only consider how, but also when to motivate additional crowd participation. This insight does not appear in the existing literature, which starts from the assumption that more crowd engagement is better.

Fourth, this research indicates that stakeholder network structure has a significant impact on OSS project performance. Specifically, this research shows that stakeholder networks featuring multiple, highly connected sub-groups tend to have stronger project management outcomes. This empirical observation lends empirical support to Levy, et al.'s~\cite{levy} micro-crowd strategy for building sustained crowd engagement. Micro-crowds encourage the formation of strong bonds between crowd members by starting with interactions between small crowds, and connecting the small crowds as the project grows. This process produces the type of hub-and-spoke networks structures that, according to the results of this research, are associated with stronger project management outcomes for OSS projects.

Finally, this study contributes to the existing body of research by validating several claims several claims within the CrowdRE literature. In addition to the bolstering the notion of micro-crowds, this research confirms that, under the right circumstances, OSS projects have the ability to absorb and benefit from a large volume of feedback from the crowd. These results further cement crowd engagement as an important construct for thinking about the requirements process.

\section{Hypotheses and Sub-Hypotheses}

This dissertation will test the following hypotheses:

\begin{itemize}
    \item The effectiveness of requirements crowdsourcing changes as network structure changes
    \item Crowdsourcing faces decreasing returns to scale. That is, due to the additional overhead of managing engagement with the crowd, OSS projects contend with a trade-off between additional crowd engagement and the ability to execute on the requirements they source from the crowd.
    \item More concentrated stakeholder networks achieve stronger project management outcomes.
\end{itemize}

The first hypothesis breaks down further into a six sub-hypotheses, one for each of the measures of effectiveness this research will consider. The sections below contain a full explanation of each hypothesis and sub-hypothesis.

\subsection{The effectiveness of requirements crowdsourcing changes with network structure}

This research begins with the hypothesis that the effectiveness of requirements crowdsourcing changes as the structure of the stakeholder network changes. To measure effectiveness, this research will develop metrics related to the advantage and disadvantages outlined in the existing literature on CrowdRE~\cite{groen}. The expectation that the effectiveness of requirements crowdsourcing would change as stakeholder network structure changes derives both from the observation in the stakeholder crosswalk research that network structure can alter patterns of influence~\cite{wood} and research from Toral, Martinez-Torres, and Barrero~\cite{toral} indicating that network structure impacts information flow in OSS projects. No study to date has empirically evaluated the joint impact of requirements crowdsourcing and stakeholder network structure on measures of effectiveness for OSS projects. In addition to the general expectation that requirements crowdsourcing and stakeholder network will have a joint effect, this research expect to see the following behavior with respect to the six project management measures of effectiveness in the data set:

\begin{itemize}
    \item Increasing the share of crowdsourced requirements increases requirement close-out time
    \item Increasing the share of crowdsourced requirements increases requirement response time
    \item Increasing the share of crowdsourced requirements increases the number of comments per requirements
    \item Increasing the share of crowdsourced requirements decreases the average number of requirements per crowd member
    \item Increasing the share of crowdsourced requirements increases requirement volume
    \item Increasing the share of crowdsourced requirements decreases the average retention time for crowd members
\end{itemize}

\subsubsection{Increasing the share of crowdsourced requirements increases requirement close-out time}

Gerth, et al.~\cite{gerth} highlight the need to organize, prioritize, and execute on a larger number of requirements as a potential drawback of CrowdRE. This suggests that, as the share of crowdsourced requirements increases, the amount of time it takes for project teams to close-out a requirement increases as well. This research anticipates that the most dramatic negative effect on task completion rate will occur at very high levels of requirements crowdsourcing, and also anticipates that increases in requirement close-out time result from a growing backlog of requirements that the project team cannot keep up with. Iyer and Lyytinen's study~\cite{iyer} suggests that higher network concentrations increase task completion velocity, meaning more highly concentrated stakeholder networks should suffer from a less dramatic increase in requirement close-out time. If increasing the share of crowdsourced requirements does result in longer requirements close-out times, it could compel project managers to seek less crowd engagement once the share of crowdsourced requirements reaches a certain threshold.

\subsubsection{Increasing the share of crowdsourced requirements increases requirement response time}

The logic for this hypothesis mirrors the argument that crowdsourcing increases requirement close-out time. Increasing the share of crowdsourced requirements means team members need to respond to more new requirements. Since responding to a requirement generally involves less time and energy than closing a requirement, an increase in the share of crowdsourced requirements may not affect requirement response time as strongly as requirement close-out time. Furthermore, since requirement response time is not tied as closely to task completion velocity and requirement close out time, stakeholder network concentration may not have as strong of an impact on response time as it does on requirement close-out time. Similarly to requirement close-out time, if increasing the share of crowdsourced requirements causes response time to increase dramatically, then there may be a point beyond which soliciting additional requirements from the crowd does more harm then good. Slow responses to crowd feedback may also itself discourage future crowd participation.

\subsubsection{Increasing the share of crowdsourced requirements increases average engagement on requirements}

The CrowdRE literature suggests~\cite{groen} that increased crowd participation results in more diverse and higher quality requirements. While difficult to measure directly, the amount of discussion a requirement generates serves as a proxy for requirement quality. This research anticipates that increasing crowd participation will increase the average amount of participation on a requirement, up to a point. If crowd members submit too many requirements, contributors and crowd members may not have the capacity to sustain a meaningful dialogue for each of them. Stakeholder network structures with stronger community formation likely promote more active engagement on requirements. Since the level of average engagement on a typical requirement can impact requirement response time and vice-versa, the study expects some level of correlation between both of these measures.


\subsubsection{Increasing the share of crowdsourced requirements decreases the average number of requirements a stakeholder generates}

Crowdsourcing expands the number of stakeholders contributing the requirements process. With no crowd participation, contributors generate every requirement, meaning that a small number stakeholders each produce a large number of requirements. This hypothesis suggests that increasing the share of crowdsourced requirements dilutes the contribution of each individual stakeholder, resulting in fewer requirements on a per stakeholder basis. The opposite could occur if crowdsourcing results in small number of crowd members producing an extremely high volume of requirements. Except in the extreme case of a small number of crowd members opening a large number of spam requirements, a higher number of average requirements per stakeholder is a positive outcome because it means that stakeholders remain engaged with the project over time.

\subsubsection{Increasing the share of crowdsourced requirements increases requirement volume}

Similar to requirement close-out and response time, this hypothesis stems from the suggestion in the literature that crowd participation results in more requirements~\cite{gerth}. A counter-intuitive result could occur, however, if project contributors produce fewer requirements to offset the increase in crowdsourced requirements. Stronger community formation likely results in higher issue volume as discussion between community members triggers ideas for new requirements. An increase in the volume of requirements is not necessarily good or bad. It is positive insofar as more requirements means better coverage over real world system uses. However, a higher volume of requirements can be detrimental if either (1) the requirements are redundant or (2) more requirements results in a growing backlog that the project team does not have the capacity to address. This study determines whether an increase in requirement volume has a positive or negative impact by observing how changes in requirement volume impact trade-offs with respect to the other measures of effectiveness. Additionally, this study  expects to find that an increase in requirement volume occurs in tandem with an increase in requirement close-out and response time, resulting in a growing backlog of unaddressed requirements. 

\subsubsection{Increased the share of crowdsourced requirements reduces crowd retention}

This research expects that, as the level of crowd participation in the requirements processes increases, retention of crowd members decreases. Intuitively, if the size of an OSS project team remains the same and they have more crowd members to engage with, each crowd member will receive less attention from the project team. Over time, if their requirements remain unaddressed, crowd members may lose interest or migrate to alternate solutions that better address their needs. Therefore, this research expects to find a negative relationship between the share of crowdsourced requirements and the retention of crowd members.

\subsection{Crowdsourcing requirements faces diminishing returns to scale}

In addition to specific hypotheses related to each measure of effectiveness, this dissertation broadly expects that requirements crowdsourcing faces decreasing returns to scale. As crowd participation increases to very high levels, the overhead required to manage the crowd offsets the advantages of requirements crowdsourcing. Therefore, as the share of crowdsourced requirements grows, projects benefit less from crowdsourcing additional requirements from the crowd. If this hypothesis holds, it suggests that project managers should work to elicit additional requirements from the crowd when the current share of crowdsourced requirements is low, and adopt automated strategies for prioritizing requirements as the share of crowdsourced requirements grows. The study will evaluate this hypothesis holistically by observing trends in the results across all six measures of effectiveness, although individual measures of effectiveness should demonstrate diminishing marginal returns to some extent as well. 

\subsection{More concentrated stakeholder network structures promote stronger project management outcomes}

The results from Iyer and Lyytinen~\cite{iyer} suggest that more concentrated stakeholder networks can more effectively handle the overhead of additional requirements. This research expects to confirm these findings. Furthermore, this research posits that more concentrated stakeholder networks face diminishing marginal returns at a higher level of requirements crowdsourcing than less concentrated networks. That is, projects with more concentrated stakeholder networks are better positioned to reap the benefits of requirements crowdsourcing because they are better able to manage the additional project management overhead that requirements crowdsourcing entails. If more concentrated stakeholder networks do result in stronger project management outcomes, project managers would benefit from implementing strategies such as micro-tasking that encourage the formation of stakeholder networks with relatively disjoint and concentrated hubs.