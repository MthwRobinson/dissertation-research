\doublespacing
\chapter{Introduction} \label{chap:intro}

The growing prevalence of systems with diverse and distributed stakeholders has stretched the limits of traditional stakeholder analysis techniques. These conditions make it difficult for project managers to identify an exhaustive list of stakeholders at the onset of a project. The consequent uncertainty during the stakeholder analysis process can negatively impact the quality of system requirements. To deal with this challenge, researchers have begun to develop techniques, known as crowd-based requirements processes (CrowdRE), that improve requirement quality by gathering feedback from crowds of stakeholders~\cite{groen}. While CrowdRE has traditionally focused on eliciting stakeholder feedback via polls and surveys prior to the formation of requirements and marketing-driven mechanisms for prioritizing existing requirements, Glinz~\cite{glinz} highlights the direct solicitation of requirements from crowds of highly-engaged stakeholders as a research gap and an opportunity for the field.

By crowdsourcing requirements, project managers can understand the range of potential use cases for a system more fully. Improved understanding of stakeholder needs translates into better system requirements and more informed prioritization decisions. Crowdsourcing, however, also has drawbacks. A high volume of requirements can overwhelm engineers and project managers~\cite{groen}. Some experts~\cite{snijders} have also questioned the quality of crowdsourced requirements. Understanding the trade-offs associated with crowdsourcing requirements enables projects managers to decide whether to invest in (1) incentivizing additional requirements generation from crowd members or (2) organizing and prioritizing existing requirements. This research suggests that the trade-off between sourcing new requirements and prioritizing existing requirements depends in part on the current share of crowdsourced requirements and the structure of the stakeholder network. In this dissertation, we use a data set that includes project management artifacts from 562 open source software (OSS) projects to empirically evaluate how stakeholder network structure and the share of crowdsourced requirements impacts key project management measures of effectiveness for OSS systems.

\section{Problem Statement}

The goal of this research is to estimate the joint effect of stakeholder network structure and the share of crowdsourced requirements on project management outcomes for OSS projects, and to determine the implications of those estimates for the application of CrowdRE techniques. While previous work, including by Wood, Sarkani, Mazzuchi, and Eveleigh~\cite{wood}, Linaker, Regnell, and Damian~\cite{linaker}, and Iyer and Lyytinen~\cite{iyer} has shown that stakeholder network structure influences prioritization decisions and task completion rates, no existing study has investigated the interaction between stakeholder network structure and requirements crowdsourcing. Moreover, the existing literature lacks robust empirical studies on the effectiveness of requirements crowdsourcing within an OSS context. This research extends the current state of the field by examining how the impact of requirements crowdsourcing changes with the share of crowdsourced requirements and the structure of the stakeholder network. Specifically, this dissertation tests the following hypotheses:

\begin{enumerate}
    \item More centralized stakeholder networks produce better outcomes for OSS projects.
    \item The effect of stakeholder network structure on OSS project outcomes varies depending on the current share of crowdsourced requirements.
    \item Requirements crowdsourcing faces diminishing marginal returns.
\end{enumerate}

This study tests these hypotheses with respect to six OSS project management measures of effectiveness: requirement close-out time, requirement response time, comment activity, the number of requirements per crowd member, the retention time for crowd members, and the volume of requirements. Requirement close-out and response time are output measures of effectiveness, while the remainder are input measures of effectiveness. To evaluate the impact of network structure, this study builds stakeholder networks by constructing a graph for each project where stakeholders are nodes, and edges represent two-way collaboration between stakeholders on a common requirement. Variables to capture stakeholder network structure include the Gini coefficient to measure network concentration, the average minimum path to measure the breadth of the network, and the clustering coefficient to measure the degree of localized clustering within the network.

To measure the joint effect of stakeholder network structure and requirements crowdsourcing, the study employs regression analysis and includes interaction terms between the network structure variables and the share of crowdsourced requirements as independent variables. The dependent variables in the study are the project management measures of effectiveness. If the hypotheses hold, the results should show stronger project management outcomes for projects with more concentrated stakeholder networks, differences in the impact of additional crowdsourcing as the structure of the stakeholder network changes, and a decrease in the value of additional requirements crowdsourcing as the share of crowdsourced requirements increases.

Statistical analysis based on a curated set of 562 open source software (OSS) projects on GitHub supports each hypothesis and concludes that OSS projects perform best when they have centralized stakeholder networks. The impact of stakeholder network structure on OSS project performance varies based on the proportion of requirements the project has already sourced from the crowd. Additionally, the study shows that the benefits of additional requirements crowdsourcing diminishes for OSS projects that already source more than 70\% of requirements from the crowd. Specifically, these projects face longer requirement close-out and response times, a growing backlog of unaddressed requirements, and worse retention for engaged crowd members.

A practical implication of this research is that OSS project managers should incentivize additional crowdsourcing when the share of crowdsourced requirements is low, and focus on triaging and prioritizing existing requirements as the share of crowdsourced requirements grows. Project managers can employ CrowdRE techniques to achieve both ends. Moreover, since more concentrated stakeholder networks promote stronger outcomes for OSS projects, project managers can benefit from adopting strategies such as fostering micro-crowds~\cite{levy} or assigning community managers to promote the formation of stakeholder networks with distinct hubs.

\section{Research Value}

This research both practical and scientific value for requirements crowdsourcing, CrowdRE, and requirements engineering more broadly. First, this research indicates that, for OSS projects, the benefit of additional crowd engagement decreases as the share of crowdsourced requirements increases. This result is interesting from a scientific perspective because (1) the existing body of research has not yet established this result and (2) the crowdsourcing literature largely assumes that more crowd engagement is a net positive. Understanding that crowdsourcing faces diminishing marginal returns will allow project mangers to utilize crowdsourcing more effectively. This research has the additional practical benefit of enabling OSS project managers to more effectively decide when to promote additional crowd engagement through methods such as gamification, and when they should focus their energy on organizing and addressing existing requirements. Specifically, project managers should promote additional crowdsourcing when the share of crowdsourced requirements is low, and transition to CrowdRE techniques for prioritizing existing requirements as the share of crowdsourced requirements rises.

Second, this research applies CrowdRE in an OSS setting, which Glinz~\cite{glinz} highlights as a gap in a 2019 review of the CrowdRE literature. The application of CrowdRE in this research is to mine user feedback in the form of GitHub issues to construct stakeholder networks. As such, this research both uses CrowdRE and studies the circumstances under which project managers should consider employing various CrowdRE techniques. Consequently, this study fortifies the usefulness of CrowdRE as both a research tool and a project management tool, and also demonstrates a novel application of CrowdRE.

Third, this research provides value by empirically validating theoretical constructs within the CrowdRE and crowd-based requirements elicitation literature~\cite{glinz}. Specifically, one of the conclusions of this research---that stakeholder networks with small, highly connected sub-groups benefit most from crowd engagement---supports Levy, Hadar, and Te'eni's~\cite{levy} claim that micro-crowds~\cite{levy} are an effective mechanism for engaging with the crowd. One implication of this research is that OSS project managers can leverage micro-crowds to encourage the formation of effective stakeholder network structures. The research also provides evidence for concerns that Groen, et al.~\cite{groen} have raised regarding the ability for project management teams to process a large volume of feedback from the crowd. The results from this research suggest that, as the share of crowdsourced requirements rises, project teams face a growing backlog of unaddressed requirements, leading to lower crowd retention and engagement. As Groen~\cite{groen} and others suggest, CrowdRE helps to mitigate these drawbacks by automatically mining crowd-generated feedback.

Finally, this research collected and analyzed the largest currently available data set of stakeholder networks for OSS projects. Previous studies have either focused on individual stakeholder networks, as in Wood, et al.~\cite{wood}, or small collections of stakeholder networks, as in Linaker, et al.~\cite{linaker} and Regnell, et al.~\cite{regnell}. While Iyer and Lyytinen~\cite{iyer} leverage a data set consisting of 259 OSS projects, it is still less than half the sample size of the 562 project data set employed in this research. Additionally, the data set that this study constructs is the first to combine stakeholder network structure and requirements crowdsourcing measures. Increasing the number of projects under consideration enables this study to draw more statistically sound conclusions about a broader range of projects.

\section{Limitations}

Before proceeding, readers should note several limitations regarding the scope of this work. First, this research considers OSS projects. Proprietary software and physical products face different challenges. Insights about crowd engagement in an OSS context do not necessarily transfer. Second, since crowd members in an OSS context are software developers, they have greater expertise than a typical crowd. A less technical crowd may lack the experience necessary to effectively generate requirements. Third, the OSS projects in the data set consist of a curated set of active, community-led projects that have not suffered substantial attrition amongst core contributors. Drawing conclusions about OSS projects more broadly would require consideration of how stakeholder network structure changes over time. Finally, since the data set only considers active OSS projects, the results do not apply to OSS projects that have not progressed beyond the system design phase. 

\section{Dissertation Structure}

This dissertation begins by reviewing the existing literature on CrowdRE, crowdsourcing, stakeholder networks, and OSS systems. The literature review identifies the following gaps in the existing body of research: 
\begin{itemize}
    \item Limited empirical research on CrowdRE
    \item Lack of CrowdRE applications within an OSS context
    \item No studies that explore the combined effect of requirements crowdsourcing and stakeholder network structure
\end{itemize}
Following the literature review section, the research contribution section explains how this dissertation addresses existing gaps within the literature and explicitly outlines the hypotheses this dissertation will test. The hypotheses this dissertation tests include:
\begin{enumerate}
    \item The effectiveness of requirements crowdsourcing changes with stakeholder network structure
    \item Requirements crowdsourcing faces diminishing marginal returns
    \item Projects with more concentrated stakeholder networks have stronger project management outcomes
\end{enumerate}
Next, the methodology chapter details the data set, explains the experimental design, justifies each modeling decision, and highlights potential drawbacks. The methodology chapter includes the methodology for building the stakeholder networks, calculating the dependent variables, and analyzing the results. This study uses generalized linear models to estimate the effect of stakeholder network structure and requirements crowdsourcing on each measure of effectiveness.

The results chapter presents the outcomes of the statistical analysis. Key observations include: 
\begin{itemize}
    \item Stakeholder network structure and requirements crowdsourcing have a joint effect on most project management outcomes
    \item Additional requirements crowdsourcing has the greatest benefit for OSS projects that currently source a limited share of requirements from the crowd
    \item For projects that source more than 70\% of requirements from the crowd, additional requirements crowdsourcing has a negative impact on project management outcomes
    \item More concentrated stakeholder networks have stronger project management outcomes
\end{itemize}

The dissertation concludes by interpreting the results, discussing limitations, and suggesting avenues for future research. As noted above, one important outcome of this study is that it indicates that different CrowdRE techniques are appropriate for OSS projects depending on the stakeholder network structure for the project and the current level of crowdsourcing. Techniques such as gamification~\cite{dalpiaz} and micro-crowds~\cite{levy} are most appropriate when the share of crowdsourced requirements is low, while automated triage and prioritization methods~\cite{lim} become more important as the share of crowdsourced requirements rises. The primary limitation of the study is that it focuses on mature OSS projects, rather early-stage OSS software, non-OSS software, or hardware. Threats to validity include that the data set does not include every measure of effectiveness of OSS software and that the study does not consider differences between various types of OSS projects or changes in the structure of stakeholder networks over time. While acknowledging these limitations, this study presents new and useful results that will help project managers employ CrowdRE more effectively within an OSS context and makes novel contributions to the CrowdRE, crowdsourcing, and stakeholder network analysis literature.